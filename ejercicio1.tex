\documentclass{article}
\usepackage{amsmath} % Paquete para fórmulas matemáticas
 \title{Definición de la función F(x) definida por partes}
\author{Tu nombre}
\date{\today}
 \begin{document}
\maketitle
 \noindent
Dada la función $F(x)$ definida por partes como:
 \[
F(x)=
\begin{cases}
    \displaystyle{\frac{1}{2}\int_((t-2)\, dt} & \text{si } 2\leq x\leq 3, \\
    \displaystyle{\frac{1}{2}\int_((2-\frac{t}{3})\, dt} & \text{si } 3< x\leq 6, \\
    0 & \text{de otro modo},
\end{cases}
\]
 podemos encontrar su forma función de distribución
 \[
F(x)=
\begin{cases}
    \displaystyle{\frac{x^{2}}{4}-x+1} & \text{si } 2\leq x\leq 3, \\
    \displaystyle{x-\frac{x^{2}}{12}-2} & \text{si } 3< x\leq 6, \\
    0 & \text{de otro modo}.
\end{cases}
\]
 Esta es la definición de la función $F(x)$ definida por partes.
 
\begin{center}
    Nota:
\end{center}
Las constante que se suman al intregrar (+1,-2) solo se realiza para visualizar
mejor la función de distribución, no afecta el valor de la función ya que:\int_
 \end{document}